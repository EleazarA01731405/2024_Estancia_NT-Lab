\documentclass[12pt]{article}

\usepackage{fancyhdr}  % Package for custom headers and footers
\usepackage{graphicx}  % Package to include images
\usepackage{array}     % Package for more flexible table columns
\usepackage{titlesec}  % Package for customizing section titles

% Set headheight and topmargin to avoid warnings
\setlength{\headheight}{60pt}
\setlength{\topmargin}{-0.5in}
\setlength{\textheight}{9in}

% Configure the header for all pages
\pagestyle{fancy}
\fancyhf{}  % Clear all header and footer fields
\fancyhead[L]{\includegraphics[height=1.5cm]{recursos_img/tec_header_logo.png}}  % Insert image in the left header
\fancyhead[R]{\parbox[b]{7cm}{\raggedleft \textbf{Comité institucional de ética en la investigación.\\ Propuesta de protocolo de investigación.}}}

\begin{document}

% Title with header
\begin{flushleft}
    \textbf{\Large Implementación de un Interfaz Cerebro-Computadora basado en P300 para el Control e Interacción de Robots}
    \par
    \textbf{15 de Agosto de 2024}
\end{flushleft}

% Start content with the header applied
\section{Investigador principal.}

\begin{table}[h!]
    \centering
    \begin{tabular}{|c|p{7cm}|} \hline
        Nombre & Javier Mauricio Antelis Ortíz \\ \hline
        Cargo & Profesor - Investigador \\ \hline
        Institución de adscripción & Tecnológico de Monterrey \\ \hline
        División a la cual pertenece & Departamento de Computación \newline Escuela de Ingeniería y Ciencias \\ \hline
        Dirección electrónica & mauricio.antelis@itesm.mx \\ \hline
        Grado máximo de estudio & Doctorado \\ \hline
        Disciplina & Ingeniería Biomédica y Computación \\ \hline
        Especialidad & Neurotecnología e Interfaces Cerebro Computador \\ \hline
    \end{tabular}
\end{table}

\begin{table}[h!]
    \centering
    \begin{tabular}{|c|p{7cm}|} \hline
        Nombre & Omar Mendoza Montoya \\ \hline
        Cargo & Profesor - Investigador \\ \hline
        Institución de adscripción & Tecnológico de Monterrey \\ \hline
        División a la cual pertenece & Departamento de Computación \newline Escuela de Ingeniería y Ciencias \\ \hline
        Dirección electrónica & omendoza83@tec.mx \\ \hline
        Grado máximo de estudio & Doctorado \\ \hline
        Disciplina & Ingeniería Biomédica y Computación \\ \hline
        Especialidad & Neurotecnología e Interfaces Cerebro Computador \\ \hline
    \end{tabular}
\end{table}

\section{Investigadores asociados.}

\begin{table}[h!]
    \centering
    \begin{tabular}{|c|c|} \hline
        Eleazar Olivas Gaspar & A01731405@tec.mx \\ \hline
        Janet Meza Hernández & A01747907@tec.mx \\ \hline
        Allan Hernández López & A01351947@tec.mx \\ \hline
        Manuel Alejandro Ramos Valdez & A00227837@tec.mx \\ \hline
        José Oswaldo Sobrevilla Vázquez & A01412742@tec.mx  \\ \hline
        Nombre & Correo \\ \hline
    \end{tabular}
\end{table}

\section{Duración del protocolo.}

\begin{table}[h!]
    \centering
    \begin{tabular}{|c|c|} \hline
        Inicio & 5 de Agosto de 2024 \\ \hline
        Fin & 26 de Nobiembre de 2024 \\ \hline
    \end{tabular}
\end{table}

\section{Tipo de experimentación.}
    Experimental y aplicativa.

\section{Línea de investigación.}
    Neurociencias, Tecnologías de la Computación, Robótica.

\section{Lugar de la investigación.}
    Laboratorio de Neurotecnología e Interfaces Cerebro-Computador (NTLab)
    Edificio del Ecosistema de Ingenieria, Arquitectura y Diseño (EIAD)

\section{Resumen.}
\section{Objetivo.}
\subsection{Hipótesis}
En el presente estudio, se propone la integración de una interfaz cerebro-computadora (BCI) para el control de dos brazos robóticos. Se evaluará si el uso de un BCI basado en el paradigma P300 puede proporcionar un control más rápido y eficiente en comparación con el seguimiento ocular a través de eye trackers. Dado que ambos métodos permiten la interacción con los brazos robóticos, se espera que el BCI basado en P300 ofrezca un tiempo de respuesta inferior, debido a su capacidad de captar directamente las señales neuronales relacionadas con la intención de movimiento, en contraposición al seguimiento ocular, que depende de movimientos físicos y tiempo de procesamiento adicional.

\section{Participantes.}
    \subsection{Tamaño de la muestra.}
        Para el protocolo de investigacion se reclutarán 30 adultos que cumplan con los criterios de inclusión.
    \subsection{Criterios de inclusión}
    \begin{itemize}
        \item Ser mayor de edad.
        \item Género y sexo indistintos.
        \item Mantener funciones cognitivas relacionadas con la atención, orientación, y memoria.
        \item No presentar trastornos musculoesqueléticos que impidan el movimiento normal.
        \item Consentimiento informado para participar en el estudio.
    \end{itemize}
        
    \subsection{Criterios de exclusión}
    \begin{itemize}
        \item Diagnóstico de enfermedades neurodegenerativas.
        \item Diagnóstico de alteraciones severas en la atención.
        \item Antecedentes de traumatismo craneoencefálico.
        \item Lesiones de nervio periférico, enfermedad vascular cerebral.
        \item Contracturas que limiten la movilidad.
        \item Condiciones que puedan introducir datos con alto nivel de ruido.
    \end{itemize}
        
\section{Material y equipo.}
\section{Descripción de procedimiento experimental.}
\section{Aspectos Éticos y de bioseguridad}
\section{Documentos complementarios.}
\end{document}
